Inteligência artificial foi definido por John McCarthy em 1955 como sendo "a ciência e engenharia de criar máquinas inteligentes" \cite{hamet2017artificial}. Existe muita literatura hoje em dia sobre IA mas ainda não existe uma opinão geral sobre os seus benefícios e malefícios \cite{hamet2017artificial}.

IA não precisa ser algo indiferenciável do comportamento ou se capaz de simular o raciocínio humano \cite{verganti2020innovation}. Nós só precisamos de um computador que seja capaz de realizar tarefas repetitivas ou que buscam um certo padrão como reconhecimento de imagem, processar um texto ou analisar grandes bancos de dados, campo hoje em dia muito dominado pelos algoritmos de máquinas \cite{verganti2020innovation, chen_artificial_2019}. Abordagens e métodos de pesquisa baseado em dados, como a aprendizagem de máquinas, são vantajosos para a análise de grandes quantidades de dados reconhecimento de padrões. Nos últimos anos, os algoritmos de aprendizagem de máquinas podem até ser usados para construir modelos a partir de dados que não estão necessariamente relacionados de forma linear \cite{edwards_if_2022}.

O conceito de inteligência artificial é muito amplo, portanto também é dividido em muitos tipos. \citeonline{li_development_2021} agrupa-os de acordo com a força, que pode ser dividida em três categorias. Uma Inteligência Artificial Fraca, que é boa em um único aspecto, pode vencer o campeão mundial de xadrez, por exemplo, mas que só pode jogar xadrez. Um nivel acima, a Inteligência Artificial Geral (AGI) contempla uma ampla gama de habilidades mentais capazes de pensar, planejar, resolver problemas, pensar abstratamente, compreender idéias complexas, aprender rapidamente e aprender com experiências anteriores. Acima, a Superinteligencia Artificial (ASI) agrupa a categoria mais forte, capaz de ser muito mais inteligente que os cérebros humanos em quase todos os campos, incluindo inovação científica, conhecimento geral e habilidades sociais. 
