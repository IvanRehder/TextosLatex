Buscas por artigos científicos foram feitas utilizando dois bancos de dados para pesquisa de artigos, o SCOPUS e o Web of Science (WOS). A pesquisa iniciou no dia 24 de outubro de 2022. A principal pergunta de pesquisa é “Como a IA auxilia no processo em métodos de criatividade dentro do desenvolvimento de produto (integrado)?”

Para as buscas os termos “Creativ*” (o asterisco no final do termo indica que palavras como "Creativity” ou "Creative” ou semelhantes devem ser aceitas) e “Artificial Intelligence” foram filtrados para o título, resumo e as palavras-chaves do artigo. Além desses termos, o termo “product development” foi filtrado ao longo do artigo todo. Por fim, apenas artigos deveriam ser indicados. 

No SCOPUS o comando de pesquisa para esse filtro é (TITLE-ABS-KEY (creativ*) AND “product  development” AND TITLE-ABS-KEY (“artificial intelligence”)) AND (LIMIT-TO (DOCTYPE, "ar")). Isso resultou em 49 artigos. A princípio, desejava-se no mínimo 50 artigos filtrados nessa primeira etapa. Como apenas 48 foram apresentados, foi adicionado o Web Of Science também na busca.

No WOS o comando para a busca foi (AB = (Creativ*) OR TI = (Creativ*) OR AK = (Creativ*)) AND (AB = ("Artificial Intelligence") OR TI = ("Artificial Intelligence") OR AK = ("Artificial Intelligence")) AND( ALL = ("Product Development")), então selecionou-se para ver apenas artigos dentre os resultados da pesquisa. Isso resultou em 4 artigos. A lista resultante após a remoção de réplicas foi de 51 artigos. 

A etapa seguinte foi a categorização em artigos “Fortes”, “Médios” e “Fracos”. Essa categorização foi feita baseada na leitura dos Resumos de todos os 51 artigos. Caso no resumo os três termos são citados, esse artigo é considerado um artigo forte. Caso apenas dois termos sejam citados, seria um artigo médio. Por fim, se uma palavra é citada, o artigo é fraco. No final tinha-se 15 artigos fortes, 17 médios e 17 fracos. Também foi desenhado categorizar os artigos por relevância ao tema, mas essa consideração provou-se redundante, já que todos os artigos fortes foram considerados de alta relevância.

Então, os 15 artigos fortes foram lidos de forma integral e de forma atenta para responder as três perguntas de pesquisa: 

\begin{itemize}

    \item Como a IA auxilia no processo em métodos de criatividade dentro do desenvolvimento de produto (integrado)?

    \item Como as tecnologias  de geração de imagem a partir de um texto usando IA auxiliam nos métodos de criatividade dentro do desenvolvimento de produto (integrado)?

    \item Como que IA enxerga as técnicas de criatividade no desenvolvimento de produto (integrado)?
\end{itemize}

Uma representação das etapas descritas acima é apresentada na Figura \ref{fig:funil_metodologia}

Buscando manter as informações organizadas, um documento foi criado para compilar todas as respostas, às perguntas de pesquisa, que cada artigo selecionado apresenta e então essas respostas foram agrupadas para então serem sintetizadas no texto apresentado no capítulo seguinte.

\begin{figure}[!htb]
    \centering
    \tikzstyle{start} = [rectangle, rounded corners, minimum width=3cm, minimum height=1.0cm,text centered, draw=black, fill=white!30, text width=3cm]
\tikzstyle{steps} = [rectangle, minimum width=3cm, minimum height=1.0cm, text centered, draw=black, fill=white!30, text width=3cm]
\tikzstyle{paral_steps} = [rectangle, minimum width=5cm, minimum height=1.5cm, text centered, draw=black, fill=white!30, text width=5cm]
    
\tikzstyle{arrow} = [ccmDBlue, rounded corners, line width = 1mm, ->] 
\tikzstyle{funil} = [black, rounded corners, dashed, line width = 0.5mm]

\begin{tikzpicture}[node distance = 3cm]
    \centering
    
    \node (start) [start] {Buscas por artigos ciêntifcos};

    \node (f1) [right of=start, xshift = 3.0cm, yshift = 0.5cm]{};
    \node (f2) [right of=start, xshift = -2.0cm, yshift = -8cm]{};
    \node (f3) [below of=f2, xshift = -0.0cm, yshift = 0.0cm]{};
    \node (f4) [left of=f3, xshift = 1.0cm, yshift = 0.0cm]{};
    \node (f5) [left of=f2, xshift = 1.0cm, yshift = 0.0cm]{};
    \node (f6) [left of=start, xshift = -3.0cm, yshift = 0.5cm]{};
    \draw[funil] (f1.center) to (f2.center);
    \draw[funil] (f2.center) to (f3.center);
    \draw[funil] (f3.center) to (f4.center);
    \draw[funil] (f4.center) to (f5.center);
    \draw[funil] (f5.center) to (f6.center);
    
    \node (scopus) [steps, below of=start, xshift = -2.5cm, yshift = 0.75cm] {SCOPUS};
    \node (wos) [steps, below of=start, yshift = 0, xshift = 2.5cm, yshift = 0.75cm] {Web of Science};
    \node (categoria) [start, below of=wos, xshift = -2.5cm, yshift = 0.5cm] {Categorização};
    \node (forte) [steps, below of=categoria, left of = categoria, xshift = -1cm, yshift = 1.0cm] {Forte};
    \node (medio) [steps, below of=categoria, yshift = 1.0cm] {Médio};
    \node (fraco) [steps, below of=categoria, right of = categoria, xshift = 1cm, yshift = 1.0cm] {Fraco};
    \node (leitura) [start, below of=medio, yshift = 0.5cm] {Leitura integral dos artigos};
    \node (pergunta) [start, below of=leitura, yshift = 1.0cm] {Repostas as perguntas de pesquisa};
    \node (escrita) [start, below of=pergunta, yshift = 1.0cm] {Escrever a sintese das respostas obtidas};
   
    \draw [arrow] (start) to ++(0,-1.0cm) to ++(-2.5cm,0) to node[midway,right]{} (scopus.north);      
    \draw [arrow] (start) to ++(0,-1.0cm) to ++(2.5cm,0) to node[midway,right]{} (wos.north);       
    \draw [arrow] (scopus) to ++(0,-1.25cm) node[text width=2.5cm, color = black, above right]{48 artigos} to ++(2.5cm,0) to node[midway,right]{} (categoria.north);
    \draw [arrow] (wos) to ++(0,-1.25cm) node[text width=2.5cm, color = black, above right]{4 artigos} to ++(-2.5cm,0) to node[text width=2.5cm, right, color = black]{51 artigos} (categoria.north);        
    \draw [arrow] (categoria) to ++(0,-1.0cm) to ++(-4.0cm,0) to node[midway,above right]{} (forte.north);       
    \draw [arrow] (categoria) to node[midway,right]{} (medio.north);
    \draw [arrow] (categoria) to ++(0,-1.0cm) to ++(4.0cm,0) to node[midway,right]{} (fraco.north);
    \draw [arrow] (forte) to ++(0,-1.25cm) node[text width=2.5cm, color = black, above right]{15 artigos} to ++(4.0cm,0) to node[midway,right]{} (leitura.north);
    \draw [arrow] (medio) to ++(0,-1.25cm) node[text width=2.5cm, color = black, above right]{17 artigos} to node[midway,right]{} (leitura.north);;
    \draw [arrow] (fraco) to ++(0,-1.25cm) node[text width=2.5cm, color = black, above right]{17 artigos} to ++(-4.0cm,0) to node[midway,right]{} (leitura.north);
    \draw [arrow] (leitura) to ++(0,-1.25cm) node[text width=2.5cm, color = black, above right]{13 artigos} to node[midway,above right]{} (pergunta.north);
    \draw [arrow] (pergunta)  to ++(0,-1.0cm) node[text width=2.5cm, color = black, above right]{} to node [midway,above right]{} (escrita.north);

\end{tikzpicture}
    \caption{Representação das etapas da metodologia}
    \label{fig:funil_metodologia}
\end{figure}