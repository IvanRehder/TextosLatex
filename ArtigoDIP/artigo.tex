\documentclass[a4paper, 10pt, dvipsnames]{article}

% Pacote de idioma
\usepackage[brazil]{babel}
% Pacote de formato de página
\usepackage[letterpaper,top=2.5cm,bottom=3cm,left=2.5cm,right=3cm,marginparwidth=1.75cm]{geometry}
% Pacote de formato de paragrafo
\usepackage[skip=5pt plus1pt, indent=40pt]{parskip}
% Pacote de formato de espaçamento entre linhas
\usepackage[onehalfspacing]{setspace}
% Pacote de hyperlinks
\usepackage[hidelinks]{hyperref}
% Pacote de referencia ABNT (ITA)
\usepackage[alf,abnt-emphasize=bf,abnt-etal-cite=2,abnt-etal-list=0,abnt-etal-text=it]{abntcite}
% Pacote que permite um trecho do texto gerar um arquivo extra
\usepackage{standalone}
%
\usepackage{array}
%
\usepackage[utf8]{inputenc}
% Pacote para ferramentas de matemática
\usepackage{amsmath}
% Pacote com opções de sublinhado (simples e duplo) e tachado
\usepackage{ulem}
% Pacote de opções para colocar imagens, legendas, sublegendas e mais opções para floats (por exemplo a opção H que força a posição)
\usepackage{graphicx, caption, subcaption, float}
% Pacote de opções para listas
\usepackage{enumitem}
% Pacotes que permitem mesclar colunas e linhas e bordas grossas nas tabelas
\usepackage{multirow, multicol, boldline}
% Pacotes do /FloatBarrier que descarrega todo os floats na fila antes de avançar
\usepackage{placeins}
% Pacotes que permite adicionar paginas de um pdf
\usepackage{pdfpages}
%\usepackage[dvipsnames]{xcolor}   % Code formating
    \definecolor{ccmBlue}{RGB}{20, 80, 200}
    \definecolor{ccmDBlue}{RGB}{25, 50, 120}
    \definecolor{ccmLBlue}{RGB}{170, 200, 230}
    \definecolor{ccmOrange}{RGB}{255, 100, 0}
    \definecolor{ccmRed}{RGB}{190, 0, 0}
    \definecolor{ccmLGray}{RGB}{210, 210, 210}
    \definecolor{ccmDGray}{RGB}{85, 85, 85}
    \definecolor{ccmWhite}{RGB}{250, 250, 250}
    
    % Ordem Paleta
    \definecolor{cor1}{RGB}{190, 000, 000}  %ccmRed
    \definecolor{cor2}{RGB}{025, 050, 120}  %ccmDBlue
    \definecolor{cor3}{RGB}{210, 210, 210}  %ccmLGray
    \definecolor{cor4}{RGB}{255, 100, 000}  %ccmOrange
    \definecolor{cor5}{RGB}{170, 200, 230}  %ccmLBlue
    \definecolor{cor6}{RGB}{085, 085, 085}  %ccmDGray
    \definecolor{cor7}{RGB}{020, 080, 200}  %ccmBlue

%Pacotes do Tikz (gerar imagens e diagramas) e salvar em .png
\usepackage{pgf, tikz}
    \usetikzlibrary{matrix,shapes.geometric,shapes.symbols,arrows.meta,positioning}
    \tikzset{>={Latex[round]}}


\title{A combinação de inteligência artificial e criatividade no desenvolvimento de produtos.}
\author{
  Cardozo, Bruno da Silva\\
  \texttt{brunobsc@ita.br}
  \and
  Machiori, Elder Lucio\\
  \texttt{eldermelchiori@gmail.com}
  \and
  Rehder, Ivan de Souza\\
  \texttt{ivan@ita.br}
}

\begin{document}
\maketitle

\begin{abstract}
Atualmente algoritmos de inteligência artificial vem auxiliando o ser humano em diversas tarefas e atividades que são repetitivas ou que demandam muito tempo. Uma dessas tarefas são as primeiras etapas de criação de um produto novo. Esse artigo busca fazer uma leitura de artigos acadêmicos a fim de entender o uso de inteligência artificial nos processos de criação dentro do desenvolvimento de produto.

Através de uma analise sistemática partindo de 51 artigos, analisou-se 7 artigos fortemente relacionados a inteligência artificial, criatividade, e desenvolvimento de produto. Por fim percebeu-se que a presença desses programas vem aumentando nas equipes de industrias e de empresas, principalmente as digitais.
\end{abstract}

\section{Introdução}
Seja para criar um novo produto, seja para solucionar um problema ou para criar novas obras de arte, o ser humano sempre contou com a sua criatividade para o auxiliar no dia-a-dia e no caminhar da humanidade, mas não significa que por estar presente o tempo todo entre os \textit{homo sapiens} não significa que todos tem uma facilidade em extrair o máximo dela. 

Existem cursos e técnicas que podem guiar uma pessoa a encontrar inspiração e assim elaborar uma resposta criativa a um problema que ela esta encarando. Recentemente, no final do século XX e no começo do século XXI, uma nova tecnologia apareceu para ajudar os seres humanos em diversas tarefas. Algoritmos especiais de computador capazes de realizar tarefas que antes exigiam tempo e concentração, agora podem ser feitas por linhas de código e em uma fração do tempo anterior. 

Esses algoritmos, classificados como algoritmos de inteligência artificial, podem também estimular a criatividade de um designer, de um engenheiro ou de um artista nas etapas preliminares de criação.

O presente artigo busca fazer um estudo no que se tem no meio acadêmico atualmente sobre a interação dessas três áreas: Inteligência artificial; Processos de criatividade; e Desenvolvimento de produto. Para isso ele propõe:

\begin{itemize}
    \item Uma revisão bibliográfica desses três conceitos, apresentados na sessão 2;
    \item Uma análise metodológica de artigos acadêmicos que unem esses 3 conceitos, conteúdo da sessão 3;
    \item O que dizem os artigos lidos, vistos aqui como os resultados e as discussões presentes na sessão 4;
    \item E por fim, as conclusões do estudo e trabalhos futuros na sessão 5.
\end{itemize}

\section{Revisão bibliográfica}

Antes de avançar nos métodos e nos resultados da análise, é importante definir os principais conceitos usados aqui nessa pesquisa.

\subsection*{Inteligência artificial (IA)}
Inteligência artificial foi definido por John McCarthy em 1955 como sendo "a ciência e engenharia de criar máquinas inteligentes" \cite{hamet2017artificial}. Existe muita literatura hoje em dia sobre IA mas ainda não existe uma opinão geral sobre os seus benefícios e malefícios \cite{hamet2017artificial}.

IA não precisa ser algo indiferenciável do comportamento ou se capaz de simular o raciocínio humano \cite{verganti2020innovation}. Nós só precisamos de um computador que seja capaz de realizar tarefas repetitivas ou que buscam um certo padrão como reconhecimento de imagem, processar um texto ou analisar grandes bancos de dados, campo hoje em dia muito dominado pelos algoritmos de máquinas \cite{verganti2020innovation, chen_artificial_2019}. Abordagens e métodos de pesquisa baseado em dados, como a aprendizagem de máquinas, são vantajosos para a análise de grandes quantidades de dados reconhecimento de padrões. Nos últimos anos, os algoritmos de aprendizagem de máquinas podem até ser usados para construir modelos a partir de dados que não estão necessariamente relacionados de forma linear \cite{edwards_if_2022}.

O conceito de inteligência artificial é muito amplo, portanto também é dividido em muitos tipos. \citeonline{li_development_2021} agrupa-os de acordo com a força, que pode ser dividida em três categorias. Uma Inteligência Artificial Fraca, que é boa em um único aspecto, pode vencer o campeão mundial de xadrez, por exemplo, mas que só pode jogar xadrez. Um nivel acima, a Inteligência Artificial Geral (AGI) contempla uma ampla gama de habilidades mentais capazes de pensar, planejar, resolver problemas, pensar abstratamente, compreender idéias complexas, aprender rapidamente e aprender com experiências anteriores. Acima, a Superinteligencia Artificial (ASI) agrupa a categoria mais forte, capaz de ser muito mais inteligente que os cérebros humanos em quase todos os campos, incluindo inovação científica, conhecimento geral e habilidades sociais. 


\subsection*{Processos de criatividade}
O uso de criatividade e a busca por inovação, são comumente vistos nas etapas preliminares de desenvolvimento de produtos, principalmente nas etapas de design. \cite{chen_artificial_2019} e são passos importantes para o surgimento de produtos, soluções e tecnologias novas\cite{edwards_if_2022}

\subsection*{Desenvolvimento de produto}
É a etapa onde são feitas as decisões que podem gerar produtos ou serviços inovadores e até mesmo testadas \cite{verganti2020innovation}. Uma das consequências de se ter um processo de desenvolvimento de produto é apresentar passos que, gradualmente, vão reduzir os riscos e as incertezas do produto, mas não ao ponto de elimina-los \cite{baxter2018product}. 

\section{Metodologia}
Buscas por artigos científicos foram feitas utilizando dois bancos de dados para pesquisa de artigos, o SCOPUS e o Web of Science (WOS). A pesquisa iniciou no dia 24 de outubro de 2022. A principal pergunta de pesquisa é “Como a IA auxilia no processo em métodos de criatividade dentro do desenvolvimento de produto (integrado)?”

Para as buscas os termos “Creativ*” (o asterisco no final do termo indica que palavras como "Creativity” ou "Creative” ou semelhantes devem ser aceitas) e “Artificial Intelligence” foram filtrados para o título, resumo e as palavras-chaves do artigo. Além desses termos, o termo “product development” foi filtrado ao longo do artigo todo. Por fim, apenas artigos deveriam ser indicados. 

No SCOPUS o comando de pesquisa para esse filtro é (TITLE-ABS-KEY (creativ*) AND “product  development” AND TITLE-ABS-KEY (“artificial intelligence”)) AND (LIMIT-TO (DOCTYPE, "ar")). Isso resultou em 49 artigos. A princípio, desejava-se no mínimo 50 artigos filtrados nessa primeira etapa. Como apenas 48 foram apresentados, foi adicionado o Web Of Science também na busca.

No WOS o comando para a busca foi (AB = (Creativ*) OR TI = (Creativ*) OR AK = (Creativ*)) AND (AB = ("Artificial Intelligence") OR TI = ("Artificial Intelligence") OR AK = ("Artificial Intelligence")) AND( ALL = ("Product Development")), então selecionou-se para ver apenas artigos dentre os resultados da pesquisa. Isso resultou em 4 artigos. A lista resultante após a remoção de réplicas foi de 51 artigos. 

A etapa seguinte foi a categorização em artigos “Fortes”, “Médios” e “Fracos”. Essa categorização foi feita baseada na leitura dos Resumos de todos os 51 artigos. Caso no resumo os três termos são citados, esse artigo é considerado um artigo forte. Caso apenas dois termos sejam citados, seria um artigo médio. Por fim, se uma palavra é citada, o artigo é fraco. No final tinha-se 15 artigos fortes, 17 médios e 17 fracos. Também foi desenhado categorizar os artigos por relevância ao tema, mas essa consideração provou-se redundante, já que todos os artigos fortes foram considerados de alta relevância.

Então, os 15 artigos fortes foram lidos de forma integral e de forma atenta para responder as três perguntas de pesquisa: 

\begin{itemize}

    \item Como a IA auxilia no processo em métodos de criatividade dentro do desenvolvimento de produto (integrado)?

    \item Como as tecnologias  de geração de imagem a partir de um texto usando IA auxiliam nos métodos de criatividade dentro do desenvolvimento de produto (integrado)?

    \item Como que IA enxerga as técnicas de criatividade no desenvolvimento de produto (integrado)?
\end{itemize}

Uma representação das etapas descritas acima é apresentada na Figura \ref{fig:funil_metodologia}

Buscando manter as informações organizadas, um documento foi criado para compilar todas as respostas, às perguntas de pesquisa, que cada artigo selecionado apresenta e então essas respostas foram agrupadas para então serem sintetizadas no texto apresentado no capítulo seguinte.

\begin{figure}[!htb]
    \centering
    \tikzstyle{start} = [rectangle, rounded corners, minimum width=3cm, minimum height=1.0cm,text centered, draw=black, fill=white!30, text width=3cm]
\tikzstyle{steps} = [rectangle, minimum width=3cm, minimum height=1.0cm, text centered, draw=black, fill=white!30, text width=3cm]
\tikzstyle{paral_steps} = [rectangle, minimum width=5cm, minimum height=1.5cm, text centered, draw=black, fill=white!30, text width=5cm]
    
\tikzstyle{arrow} = [ccmDBlue, rounded corners, line width = 1mm, ->] 
\tikzstyle{funil} = [black, rounded corners, dashed, line width = 0.5mm]

\begin{tikzpicture}[node distance = 3cm]
    \centering
    
    \node (start) [start] {Buscas por artigos ciêntifcos};

    \node (f1) [right of=start, xshift = 3.0cm, yshift = 0.5cm]{};
    \node (f2) [right of=start, xshift = -2.0cm, yshift = -8cm]{};
    \node (f3) [below of=f2, xshift = -0.0cm, yshift = 0.0cm]{};
    \node (f4) [left of=f3, xshift = 1.0cm, yshift = 0.0cm]{};
    \node (f5) [left of=f2, xshift = 1.0cm, yshift = 0.0cm]{};
    \node (f6) [left of=start, xshift = -3.0cm, yshift = 0.5cm]{};
    \draw[funil] (f1.center) to (f2.center);
    \draw[funil] (f2.center) to (f3.center);
    \draw[funil] (f3.center) to (f4.center);
    \draw[funil] (f4.center) to (f5.center);
    \draw[funil] (f5.center) to (f6.center);
    
    \node (scopus) [steps, below of=start, xshift = -2.5cm, yshift = 0.75cm] {SCOPUS};
    \node (wos) [steps, below of=start, yshift = 0, xshift = 2.5cm, yshift = 0.75cm] {Web of Science};
    \node (categoria) [start, below of=wos, xshift = -2.5cm, yshift = 0.5cm] {Categorização};
    \node (forte) [steps, below of=categoria, left of = categoria, xshift = -1cm, yshift = 1.0cm] {Forte};
    \node (medio) [steps, below of=categoria, yshift = 1.0cm] {Médio};
    \node (fraco) [steps, below of=categoria, right of = categoria, xshift = 1cm, yshift = 1.0cm] {Fraco};
    \node (leitura) [start, below of=medio, yshift = 0.5cm] {Leitura integral dos artigos};
    \node (pergunta) [start, below of=leitura, yshift = 1.0cm] {Repostas as perguntas de pesquisa};
    \node (escrita) [start, below of=pergunta, yshift = 1.0cm] {Escrever a sintese das respostas obtidas};
   
    \draw [arrow] (start) to ++(0,-1.0cm) to ++(-2.5cm,0) to node[midway,right]{} (scopus.north);      
    \draw [arrow] (start) to ++(0,-1.0cm) to ++(2.5cm,0) to node[midway,right]{} (wos.north);       
    \draw [arrow] (scopus) to ++(0,-1.25cm) node[text width=2.5cm, color = black, above right]{48 artigos} to ++(2.5cm,0) to node[midway,right]{} (categoria.north);
    \draw [arrow] (wos) to ++(0,-1.25cm) node[text width=2.5cm, color = black, above right]{4 artigos} to ++(-2.5cm,0) to node[text width=2.5cm, right, color = black]{51 artigos} (categoria.north);        
    \draw [arrow] (categoria) to ++(0,-1.0cm) to ++(-4.0cm,0) to node[midway,above right]{} (forte.north);       
    \draw [arrow] (categoria) to node[midway,right]{} (medio.north);
    \draw [arrow] (categoria) to ++(0,-1.0cm) to ++(4.0cm,0) to node[midway,right]{} (fraco.north);
    \draw [arrow] (forte) to ++(0,-1.25cm) node[text width=2.5cm, color = black, above right]{15 artigos} to ++(4.0cm,0) to node[midway,right]{} (leitura.north);
    \draw [arrow] (medio) to ++(0,-1.25cm) node[text width=2.5cm, color = black, above right]{17 artigos} to node[midway,right]{} (leitura.north);;
    \draw [arrow] (fraco) to ++(0,-1.25cm) node[text width=2.5cm, color = black, above right]{17 artigos} to ++(-4.0cm,0) to node[midway,right]{} (leitura.north);
    \draw [arrow] (leitura) to ++(0,-1.25cm) node[text width=2.5cm, color = black, above right]{13 artigos} to node[midway,above right]{} (pergunta.north);
    \draw [arrow] (pergunta)  to ++(0,-1.0cm) node[text width=2.5cm, color = black, above right]{} to node [midway,above right]{} (escrita.north);

\end{tikzpicture}
    \caption{Representação das etapas da metodologia}
    \label{fig:funil_metodologia}
\end{figure}

\section{Resultados e discussões}
Como dito anteriormente, a pergunta que buscou-se responder foi "Como a inteligência artificial auxilia no processo em métodos de criatividade dentro do desenvolvimento de produto?"

\subsection*{Processamento de linguagem natural e processament de imagem}

Ao longo da leitura dos artigos é possível perceber que alguns dos mecanismo utilizado nas aplicações de IA para estimular a criatividade são o processamento de linguagem natural (PLN) e o processamento de imagem. \citeonline{chen_artificial_2019} e \citeonline{he_mining_2019} utilizaram ambos nas suas soluções.
% Linguagem
% Imagem

\citeonline{chen_artificial_2019} propõem um método para estimular a criatividade que pode ser dividido em duas partes: uma usando processamento natural de linguagem; e outra utilizando processamento de imagens. Os autores testaram esse método contra um grupo de controle que utilizava apenas o google e brainstorming para estimular a criatividade e o resultado foi que o método desenvolvido não só gerou mais ideias quanto gerou mais ideias originais.
% -------------------------------------------------------------
% -------------------------------------------------------------
% Linguagem
% Grafos
\citeonline{he_mining_2019} elaboram métodos para estimular a criatividade usando um processamento de linguagem natural e teoria de grafos para gerar uma nuvem de palavras baseado num banco de dados criado com textos de ideias de projetos de soluções de problemas. Essa nuvem de palavras destaca as palavras mais utilizadas nesses projetos. Logo, se os designers buscam encontrar uma solução inovadora, deve-se procurar por palavras menores, ou seja, menos citadas. Uma segunda etapa dessa metodologia é usar um algoritmo para selecionar as palavras em busca de criar combinações que possam gerar uma ideia.
% -------------------------------------------------------------
% -------------------------------------------------------------


% *************************************************************
% *************************************************************
% Linguagem (Fuzzy)
% Delphi

\subsection*{Tomada de decisão}

\citeonline{hsueh_supporting_2022} utilizaram a ferramenta Delphi para obter critérios de desenvolvimento de produtos culturais e criativos. Foram selecionados 4 critérios e implementados junto a 81 cenários em um algoritmo de tomada de decisão multiatributo (DFuzzy) que pode ser implementado através de valores de inputs que podem ser qualidades quantitativas em diferentes unidades ou semântica difusa (Fuzzy Semantics). O algoritmo é capaz de atribuir resultados quantitativos para cada critério em cada cenário utilizando um modelo de IA de cálculo, para auxiliar no cálculo quantitativo final de cada um dos cenários. Possibilitando assim auxiliar no processo criativo através dos valores de cada cenário e auxiliando na tomada de decisão nas fases de projeto.

The DFuzzy model is a multiattribute decision-making model capable of both qualitative and quantitative analysis. It features a highly objective and scientifc calculation function and is adaptive to the input units of diferent criteria at the same time. The units of the four criteria were defined as points, percentages, and items. The DFuzzy model also implements imprecise semantics (e.g., high, medium, and low) for quantitative calculations, which cannot be attained through conventional mathematic models
% -------------------------------------------------------------
% -------------------------------------------------------------


% *************************************************************
% *************************************************************
\subsection*{Outras análises}

% Meta-Analise
\citeonline{li_development_2021} baseou-se na comparação entre padrões tradicionais de desenvolvimento utilizados no design de produtos e compreender os novos padrões de design auxiliados pela inteligência artificial, de modo a alcançar mais rapidamente o objetivo de simplificação de processos e inovação no design.

De forma geral, segundo \citeonline{li_development_2021}, as tecnologias de inteligência artificial podem substituir e auxiliar os designers durante o processo de desenvolvimento. Atualmente, a IA ainda não mudou completamente a maneira com que os produtos são desenvolvidos, mas tem inovado dentro das metodologias de desenvolvimento integrado de produto. O estudo demonstra que, na etapa informacional (fase de pesquisa) do design de produto, a IA interage de forma com que não só seja capaz realizar a pesquisa de produtos similares anteriores, mas também de julgar as vantagens e desvantagens entre diferentes conceitos. No design thinking, \citeonline{li_development_2021} esclarece que os requisitos do projeto do produto sejam constantemente atualizados e iterativos com a presença de tecnologias de IA. Neste caso, as soluções trazidas por essas tecnologias podem se tornar o ponto de partida de uma atualização do produto em uma posterior fase de validação. A avaliação e verificação dos resultados do projeto do produto após o ajuste pode tornar o feedback sobre o projeto do produto mais preciso e mais propício para o aperfeiçoamento do produto pelos designers. No processo de design tradicional, esta etapa é entediante e demorada, quando existente.

Em suma, a inteligência artificial cria ideias de design mais eficazes no desenvolvimento de produtos, de acordo com \citeonline{li_development_2021}. Atualmente, a IA penetra gradualmente no criação de novos produtos, trazendo uma influência para a vida cotidiana. Portanto, é esperado que as tecnologias de IA sejam capazes de integrar todo o processo de concepção de produto.
% -------------------------------------------------------------
% -------------------------------------------------------------

\section{Conclusões}
As tecnologias de inteligência artificial estão sendo refinadas, crescendo cada vez mais e se fazendo presente em várias etapas do desenvolvimento de produto. Neste estudo, notou-se que as técnicas que utilizaram o processamento de linguagem natural e processamento de imagem, especialmente, auxiliaram nos métodos de criatividade durante a fase de ideação com resultados surpreendentes. Ademais, técnicas de IA também foram amplamente utilizadas em situações de tomada de decisão, onde a capacidade de avaliação de cenários e de quantificação de atributos subjetivos auxiliou no processo criativo durante as fases de desenvolvimento de produtos ou estimulando estratégias diferentes para solução de problemas.

Entende-se que a inteligência artificial baseada em geração de texto-para-imagem pode ser capaz de complementar os atuais métodos de criatividade, contribuindo para o desenvolvimento de produtos inovadores. Visando trabalhos futuros, pretende-se investigar o possível uso de técnicas de processamento de imagem, especialmente as baseadas no aprendizado de conceitos visuais com supervisão da linguagem natural, nas etapas iniciais do desenvolvimento de produto. 


%\bibliographystyle{abntex2-alf}
\bibliography{referencias}

\end{document}