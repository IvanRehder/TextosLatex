Seja para criar um novo produto, seja para solucionar um problema ou para criar novas obras de arte, o ser humano sempre contou com a sua criatividade para o auxiliar no dia-a-dia e no caminhar da humanidade, mas não significa que por estar presente o tempo todo entre os \textit{homo sapiens} não quer dizer que todos tem uma facilidade em extrair o máximo dela. 

Existem cursos e técnicas que podem guiar uma pessoa a encontrar inspiração e assim elaborar uma resposta criativa a um problema que ela esta encarando. Recentemente, no final do século XX e no começo do século XXI, uma nova tecnologia apareceu para ajudar os seres humanos em diversas tarefas. Algoritmos especiais de computador capazes de realizar tarefas que antes exigiam tempo e concentração, agora podem ser feitas por linhas de código e em uma fração do tempo anterior. 

Esses algoritmos, classificados como algoritmos de inteligência artificial, podem também estimular a criatividade de um designer, de um engenheiro ou de um artista nas etapas preliminares de criação.

O presente artigo busca fazer um estudo no que se tem no meio acadêmico atualmente sobre a interação dessas três áreas: Inteligência artificial; Processos de criatividade; e Desenvolvimento de produto. Para isso ele propõe:

\begin{itemize}
    \item Uma revisão bibliográfica desses três conceitos, apresentados na sessão 2;
    \item Um análise metodológica de artigos acadêmicos que unem esses 3 conceitos, conteúdo da sessão 3;
    \item O que dizem os artigos lidos, vistos aqui como os resultados e as discussões presentes na sessão 4;
    \item E por fim, as conclusões do estudo na sessão 5.
\end{itemize}