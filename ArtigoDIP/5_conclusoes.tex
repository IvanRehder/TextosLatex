As tecnologias de inteligência artificial estão sendo refinadas, crescendo cada vez mais e se fazendo presente em várias etapas do desenvolvimento de produto. Neste estudo, notou-se que as técnicas que utilizaram o processamento de linguagem natural e processamento de imagem, especialmente, auxiliaram nos métodos de criatividade durante a fase de ideação com resultados surpreendentes. Ademais, técnicas de IA também foram amplamente utilizadas em situações de tomada de decisão, onde a capacidade de avaliação de cenários e de quantificação de atributos subjetivos auxiliou no processo criativo durante as fases de desenvolvimento de produtos ou estimulando estratégias diferentes para solução de problemas.

Entende-se que a inteligência artificial baseada em geração de texto-para-imagem pode ser capaz de complementar os atuais métodos de criatividade, contribuindo para o desenvolvimento de produtos inovadores. Visando trabalhos futuros, pretende-se investigar o possível uso de técnicas de processamento de imagem, especialmente as baseadas no aprendizado de conceitos visuais com supervisão da linguagem natural, nas etapas iniciais do desenvolvimento de produto. 
