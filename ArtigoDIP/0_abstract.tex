Atualmente algoritmos de inteligência artificial vem auxiliando o ser humano em diversas tarefas e atividades que são repetitivas ou que demandam muito tempo. Uma dessas tarefas são as primeiras etapas de criação de um produto novo. Esse artigo busca fazer uma leitura de artigos acadêmicos a fim de entender o uso de inteligência artificial nos processos de criação dentro do desenvolvimento de produto.

Através de uma analise sistemática partindo de 51 artigos, analisou-se 7 artigos fortemente relacionados a inteligência artificial, criatividade, e desenvolvimento de produto. Por fim percebeu-se que a presença desses programas vem aumentando nas equipes de industrias e de empresas, principalmente as digitais.