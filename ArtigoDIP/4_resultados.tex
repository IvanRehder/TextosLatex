Como dito anteriormente, a pergunta que buscou-se responder foi "Como a inteligência artificial auxilia no processo em métodos de criatividade dentro do desenvolvimento de produto?"

\subsection*{Processamento de linguagem natural e processamento de imagem}

% Linguagem
% Imagem
Ao longo da leitura dos artigos é possível perceber que alguns dos mecanismo utilizado nas aplicações de IA para estimular a criatividade são o processamento de linguagem natural (PLN) e o processamento de imagem. \citeonline{chen_artificial_2019} e \citeonline{he_mining_2019} utilizaram ambos nas suas soluções.

% Linguagem
% Imagem
\citeonline{chen_artificial_2019} propõem um método para estimular a criatividade que pode ser dividido em duas partes: uma usando processamento natural de linguagem; e outra utilizando processamento de imagens. Os autores testaram esse método contra um grupo de controle que utilizava apenas o google e brainstorming para estimular a criatividade e o resultado foi que o método desenvolvido não só gerou mais ideias quanto gerou mais ideias originais.
% -------------------------------------------------------------
% -------------------------------------------------------------

% Linguagem
% Grafos
\citeonline{he_mining_2019} elaboram métodos para estimular a criatividade usando um processamento de linguagem natural e teoria de grafos para gerar uma nuvem de palavras baseado num banco de dados criado com textos de ideias de projetos de soluções de problemas. Essa nuvem de palavras destaca as palavras mais utilizadas nesses projetos. Logo, se os designers buscam encontrar uma solução inovadora, deve-se procurar por palavras menores, ou seja, menos citadas. Uma segunda etapa dessa metodologia é usar um algoritmo para selecionar as palavras em busca de criar combinações que possam gerar uma ideia.

% -------------------------------------------------------------
% -------------------------------------------------------------


% *************************************************************
% *************************************************************
% Linguagem (Fuzzy)
% Delphi

\subsection*{Tomada de decisão}

\citeonline{hsueh_supporting_2022} Eles fizeram um Delphi, um questionário para identificar atributos chave sem um determinado problema. Eles solictar os atributos mais importantes nas tomadas de decisões para 17 profissionais da área, chegando a 4 critérios importantes. 

A técnica Fuzzy é usada junto com esses critérios para quantificar uma tomada de decisão. A técnica Fuzzy é uma maneira de quantificar parâmetros qualitativos e é ela quem permite pontuar os critérios e assim facilitar a tomada de decisão.

O uso desses dois métodos em conjunto, Delphi e Fuzzy, gerou o DFuzzy, objeto de estudo do artigo. O DFuzzy foi testado em 81 cenários diferentes e os 3 cenários com maiores pontuações foram 3 ideias que já tinha recebido títulos de ideias mais inovadoras dos anos de 2015 e 2017 da Malasia.


% -------------------------------------------------------------
% -------------------------------------------------------------


% *************************************************************
% *************************************************************
\subsection*{Outras análises}

% Meta-Analise
\citeonline{li_development_2021} baseou-se na comparação entre padrões tradicionais de desenvolvimento utilizados no design de produtos e compreender os novos padrões de design auxiliados pela inteligência artificial, de modo a alcançar mais rapidamente o objetivo de simplificação de processos e inovação no design.

De forma geral, segundo \citeonline{li_development_2021}, as tecnologias de inteligência artificial podem substituir e auxiliar os designers durante o processo de desenvolvimento. Atualmente, a IA ainda não mudou completamente a maneira com que os produtos são desenvolvidos, mas tem inovado dentro das metodologias de desenvolvimento integrado de produto. O estudo demonstra que, na etapa informacional (fase de pesquisa) do design de produto, a IA interage de forma com que não só seja capaz realizar a pesquisa de produtos similares anteriores, mas também de julgar as vantagens e desvantagens entre diferentes conceitos. No design thinking, \citeonline{li_development_2021} esclarece que os requisitos do projeto do produto sejam constantemente atualizados e iterativos com a presença de tecnologias de IA. Neste caso, as soluções trazidas por essas tecnologias podem se tornar o ponto de partida de uma atualização do produto em uma posterior fase de validação. A avaliação e verificação dos resultados do projeto do produto após o ajuste pode tornar o feedback sobre o projeto do produto mais preciso e mais propício para o aperfeiçoamento do produto pelos designers. No processo de design tradicional, esta etapa é entediante e demorada, quando existente.

Em suma, a inteligência artificial cria ideias de design mais eficazes no desenvolvimento de produtos, de acordo com \citeonline{li_development_2021}. Atualmente, a IA penetra gradualmente no criação de novos produtos, trazendo uma influência para a vida cotidiana. Portanto, é esperado que as tecnologias de IA sejam capazes de integrar todo o processo de concepção de produto.
% -------------------------------------------------------------
% -------------------------------------------------------------