Como dito anteriormente, a pergunta que buscou-se responder foi "Como a inteligência artificial (IA) auxilia no processo em métodos de criatividade dentro do desenvolvimento de produto?"

\subsection*{Processamento de linguagem natural e processamento de imagem}

% Linguagem
% Imagem
Ao longo da leitura dos artigos é possível perceber que alguns dos mecanismo utilizado nas aplicações de IA para estimular a criatividade são o processamento de linguagem natural (PLN) e o processamento de imagem. \citeonline{chen_artificial_2019} e \citeonline{he_mining_2019} utilizaram ambos nas suas soluções.

% Linguagem
% Imagem
\citeonline{chen_artificial_2019} propõem um método baseado em dados para estimular a criatividade durante a etapa de ideação no desenvolvimento de produtos. O método foi dividido em duas partes: a primeira, usando processamento natural de linguagem; a segunda, utilizando processamento de imagens. Os autores testaram o método contra um grupo de controle que utilizava apenas o google e brainstorming para estimular a criatividade. Neste cenário, o resultado foi que o método desenvolvido auxiliou numa maior geração de ideias e, também, ideias mais originais.

O primeiro método se baseia no conceito de rede semântica, que é uma rede associativa artificial entre entidades semânticas (objetos, conceitos, ideias específicas, produtos) ligadas por relações semânticas, de forma a ser capaz de quantificar a diversidade dos diversos atributos. De forma geral, o autor explica que as fontes de dados  para o treino podem ser obtidas de várias maneiras, tais como a raspagem web. Para a etapa de pré-processamento, Chen discute que as abordagens linguísticas são melhores na captura de significados semânticos e sintáticos no processamento de linguagem natural. Neste caso, métodos supervisionados têm mostrado melhor desempenho beneficiando-se de estruturas avançadas de aprendizagem de máquinas. A partir disso, o algoritmo desenvolvido baseou-se em permitir a interação do usuario com a IA de forma com que seja possível avançar ou recuar pelas relações semânticas. Isto é, a criatividade é estimulada a partir da possibilidade de caminhar entre entidades semânticas associadas, explícitas ou implicitamente, pela IA.

No método utilizado para o processamento de imagens, o estudo investigou como produzir imagens que sintetizam características importantes a partir de dois conceitos, enquadrando-se em duas categorias distintas usando a formulação de redes adversárias generativas (GAN). Como método de aprendizado supervisionado, o autor estabeleceu dois conjuntos de dados através da coleta de imagens que representavam dois conceitos conhecidos, aplicando as Redes Adversárias Generativas Convolutivas Profundas (DCGANs).

% -------------------------------------------------------------
% -------------------------------------------------------------

% Linguagem
% Grafos
\citeonline{he_mining_2019} elaboram métodos para estimular a criatividade usando um processamento de linguagem natural e teoria de grafos para gerar uma nuvem de palavras baseado num banco de dados criado com textos de ideias de projetos de soluções de problemas. Essa nuvem de palavras destaca as palavras mais utilizadas nesses projetos. Logo, se os designers buscam encontrar uma solução inovadora, deve-se procurar por palavras menores, ou seja, menos citadas. Uma segunda etapa dessa metodologia é usar um algoritmo para selecionar as palavras em busca de criar combinações que possam gerar uma ideia.

% -------------------------------------------------------------
% -------------------------------------------------------------

Esses métodos mostram como o seres humanos podem ter a sua criatividade estimulada apenas com estimulos visuais.

% *************************************************************
% *************************************************************
% Linguagem (Fuzzy)
% Delphi

\subsection*{Inteligência artificial auxiliando tomada de decisão}

Alguns algoritimos de IA facilitam a tomada de decisão de uma equipe de criação. \citeonline{hsueh_supporting_2022} desenvolveram uma ferramenta para facilitar a tomada de decisão e \citeonline{verganti2020innovation} analisou o uso de IA em três grandes empresas e em uma dessas analises ele mostra como foi utilizado para definir os parametros de criação de novos produtos. Semelhantemente, \citeonline{buhamdan2022use} desenveolveu um algoritmo para definir a melhor estratégia para correções em obras civis.

\citeonline{hsueh_supporting_2022} fizeram um Delphi, um questionário para identificar atributos chave sem um determinado problema. Eles solicitaram os atributos mais importantes nas tomadas de decisões para 17 profissionais da área, chegando a 4 critérios importantes. 

A técnica Fuzzy é usada junto com esses critérios para quantificar uma tomada de decisão. A técnica Fuzzy é uma maneira de quantificar parâmetros qualitativos e é ela quem permite pontuar os critérios e assim facilitar a tomada de decisão.

O uso desses dois métodos em conjunto, Delphi e Fuzzy, gerou o DFuzzy, objeto de estudo do artigo. O DFuzzy foi testado em 81 cenários diferentes e os 3 cenários com maiores pontuações foram 3 ideias que já tinham recebido títulos de ideias mais inovadoras dos anos de 2015 e 2017 da Malasia.

% -------------------------------------------------------------
% -------------------------------------------------------------

\citeonline{de2020artificial} utilizaram um sistema baseado em conhecimento usando IA para auxiliar o gerenciamento de conhecimento na seleção de técnicas de criatividade e inovação. Os autores identificaram que os especialistas conheciam menos da metade das técnicas apresentadas no estudo, mostrando uma diferença de conhecimento sobre as técnicas dentre os especialistas 

As técnicas de estimulo a criatividade tornaram-se importantes adições para aprimorar o processo criativo. No entanto o elevado número de técnicas existentes, o conhecimento necessário para selecioná-las e utilizá-las, torna-se necessária a utilização de IA. Esse programa de IA é capaz de reter o conhecimento sobre as técnicas e também sobre como, quando e por que usar cada uma \cite{de2020artificial}.

% -------------------------------------------------------------
% -------------------------------------------------------------

\citeonline{verganti2020innovation} realizou estudos de casos onde a IA auxiliou em diferentes fases do desenvolvimento de produto. Dentre eles o caso da Netflix que utilizou os dados dos usuários e sua atividade na plataforma para servir de entrada para seus algoritmos de IA. Estes algoritmos foram implementados através de três abordagens: 

\begin{itemize}
    \item \textit{Supervised Learning},
    utilizando dados de usuários considerados similares, para calibrar as escolhas de recomendações de conteúdo
    
    \item \textit{Unsupervised Learning},
    utilizando dados para predizer qual conteúdo poderia ser criado, este conteúdo foi utilizado para ajudar a avaliar o potencial da série \textit{House of Cards}.

    \item \textit{Reinforced Learning},
    utilizado para explorar opções e também para explorar soluções envolvendo a escolha das artes de cada série a ser apresentada para cada um dos usuários individualmente.
\end{itemize}

% -------------------------------------------------------------
% -------------------------------------------------------------

\citeonline{buhamdan2022use} elaboraram um código em Python que, utilizando uma estratégia de aprendizado reinforçado para encontrar a melhor solução para um problema de design em um modelo de um prédio com aproximadamente 8500 m². O seu programa busca otimizar uma função recompensa, assim reduzindo o custo, permitindo que o programa faça interações partindo de uma estado inicial até o valor convergir.

% -------------------------------------------------------------
% -------------------------------------------------------------

Esses artigos mostram que a IA também pode ser utilizadas para facilitar na tomada de decisão de criação de novos produtos e estratégias de soluções de problemas.

% *************************************************************
% *************************************************************
\subsection*{Outras análises do uso de IA no desenvolvimento de produto}

% Meta-Analise
\citeonline{li_development_2021} baseou-se na comparação entre padrões tradicionais de desenvolvimento utilizados no design de produtos e compreender os novos padrões de design auxiliados pela IA, de modo a alcançar mais rapidamente o objetivo de simplificação de processos e inovação no design.

De forma geral, segundo \citeonline{li_development_2021}, as tecnologias de IA podem substituir e auxiliar os designers durante o processo de desenvolvimento. Atualmente, a IA ainda não mudou completamente a maneira com que os produtos são desenvolvidos, mas tem inovado dentro das metodologias de desenvolvimento integrado de produto. O estudo demonstra que, na etapa informacional (fase de pesquisa) do design de produto, a IA interage de forma com que não só seja capaz realizar a pesquisa de produtos similares anteriores, mas também de julgar as vantagens e desvantagens entre diferentes conceitos. No design thinking, \citeonline{li_development_2021} esclarece que os requisitos do projeto do produto sejam constantemente atualizados e iterativos com a presença de tecnologias de IA. Neste caso, as soluções trazidas por essas tecnologias podem se tornar o ponto de partida de uma atualização do produto em uma posterior fase de validação. A avaliação e verificação dos resultados do projeto do produto após o ajuste pode tornar o feedback sobre o projeto do produto mais preciso e mais propício para o aperfeiçoamento do produto pelos designers. No processo de design tradicional, esta etapa é entediante e demorada, quando existente.

Em suma, a IA cria ideias de design mais eficazes no desenvolvimento de produtos, de acordo com \citeonline{li_development_2021}. Atualmente, a IA penetra gradualmente no criação de novos produtos, trazendo uma influência para a vida cotidiana. Portanto, é esperado que as tecnologias de IA sejam capazes de integrar todo o processo de concepção de produto.
% -------------------------------------------------------------
% -------------------------------------------------------------